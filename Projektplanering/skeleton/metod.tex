\section{Work process}
The process of developing the proof editor will be split into three main phases: background research, development and testing.

\subsection{Researching current editors}
In the research phase, we will look at existing proof editors and assistants available online. We will test both editors which come with a GUI and text based editors to identify both important features and things to avoid. Key aspects in the evaluation of different apps will include efficiency, ease of use, installation process and the level of assistance provided by the app in the process of constructing a proof.

\subsection{Developing the editor}
Based on our experience from testing the currently available editors, we will start the development process by making some initial design choices for our editor. These initial choices will include choosing a user interface type and deciding how to distribute the application to the users. Once we have outlined the features of the application, we will decide what technology stack to use.

During the development, we intend to use a Kanban board to keep track of which tasks we are currently working on, which tasks are coming up next, and which tasks are done. One member of the team will be responsible for prioritising the upcoming tasks and each member of the team will pick tasks from the board to work on.

Coordination of the team will be managed through group meetings, occuring twice every week, where current issues will be discussed.

\subsection{Testing, evaluating and tweaking}
Testing of the proof editor will be carried out using a group of test users who match our target audience. These test users will be given a set of problems, similar to exam problems used in university courses in logic.

During the test, the users will be asked to solve some problems using nothing but pen and paper and other problems using the proof editor. By comparing the time it takes to produce a correct proof using the editor with the time it takes using pen and paper we can determine if the editor allows for a more efficient process than regular pen and paper.